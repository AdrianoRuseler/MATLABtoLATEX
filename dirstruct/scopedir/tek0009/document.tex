\documentclass{minimal} 
\usepackage{tikz}
\usetikzlibrary{shapes,decorations}
\usepackage{amsmath,amssymb}
\begin{document}

% We need to save the node
% Every append after command might be useful to have this code
\def\savelastnode{\pgfextra\edef\tmpA{\tikzlastnode}\endpgfextra}
\def\restorelastnode{\pgfextra\edef\tikzlastnode{\tmpA}\endpgfextra}
% Define box and box title style
\tikzstyle{mybox} = [draw=red, fill=blue!20, very thick,
    rectangle, rounded corners, inner sep=10pt, inner ysep=20pt]
\tikzstyle{fancytitle} =[fill=red, text=white]
\tikzstyle{club suit} = [append after command={%
    \savelastnode node[fancytitle,rounded corners] at (\tikzlastnode.east)%
    {$\clubsuit$}\restorelastnode }]
\tikzstyle{title} = [append after command={%
    \savelastnode node[fancytitle,right=10pt] at (\tikzlastnode.north west)%
    {#1}\restorelastnode}]

\begin{tikzpicture}
\node [mybox,club suit,title=Fancy title] (box){%
    \begin{minipage}{0.50\textwidth}
        To calculate the horizontal position the kinematic differential
        equations are needed:
        \begin{align}
            \dot{n} &= u\cos\psi -v\sin\psi \\
            \dot{e} &= u\sin\psi + v\cos\psi
        \end{align}
        For small angles the following approximation can be used:
        \begin{align}
            \dot{n} &= u -v\delta_\psi \\
            \dot{e} &= u\delta_\psi + v
        \end{align}
    \end{minipage}
};
% These are no longer needed! It is now situated in the (box) node command!
%\node[fancytitle, right=10pt] at (box.north west) {A fancy title};
%\node[fancytitle, rounded corners] at (box.east) {$\clubsuit$};
\end{tikzpicture}%
\end{document}

