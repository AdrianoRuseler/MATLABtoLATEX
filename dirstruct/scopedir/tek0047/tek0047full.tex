%
\documentclass{standalone}
% \usepackage{tikz}
\usepackage{pgfplots} % loads also tikz
\usepackage{grffile}
\pgfplotsset{compat=newest} % This will configure the compatibility layer
\usetikzlibrary{plotmarks}
\usepackage{amsmath}
\usepackage{color} % Using this package, you can set the font color, text background, or page background
\usetikzlibrary{arrows,shapes,positioning}
\usepackage{anyfontsize}
\usepackage{siunitx}

\tikzset{ % Set line width
	ultra thin/.style= {line width=0.1pt},
	very thin/.style=  {line width=0.2pt},
	thin/.style=       {line width=0.4pt},% thin is the default
	semithick/.style=  {line width=0.6pt},
	thick/.style=      {line width=0.8pt},
	very thick/.style= {line width=1.2pt},
	ultra thick/.style={line width=1.6pt}
}

% Defines color spec 
% http://www.color-hex.com for color details
\definecolor{dallcolor}{HTML}{303030} % Gray19
\definecolor{ch1color}{HTML}{0000D6}
\definecolor{ch2color}{HTML}{009D9A}
\definecolor{ch3color}{HTML}{E600E6}
\definecolor{ch4color}{HTML}{18A618}
\definecolor{mathcolor}{HTML}{F70000}
\definecolor{ref1color}{HTML}{0000FF}
\definecolor{ref2color}{HTML}{0000FF}
\definecolor{ref3color}{HTML}{0000FF}
\definecolor{ref4color}{HTML}{0000FF}
\definecolor{d0color}{HTML}{2b2b2b}
\definecolor{d1color}{HTML}{2b2b2b}
\definecolor{d2color}{HTML}{2b2b2b}
\definecolor{d3color}{HTML}{2b2b2b}
\definecolor{d4color}{HTML}{2b2b2b}
\definecolor{d5color}{HTML}{2b2b2b}
\definecolor{d6color}{HTML}{2b2b2b}
\definecolor{d7color}{HTML}{2b2b2b}
\definecolor{d8color}{HTML}{2b2b2b}
\definecolor{d9color}{HTML}{2b2b2b}
\definecolor{d10color}{HTML}{2b2b2b}
\definecolor{d11color}{HTML}{2b2b2b}
\definecolor{d12color}{HTML}{2b2b2b}
\definecolor{d13color}{HTML}{2b2b2b}
\definecolor{d14color}{HTML}{2b2b2b}
\definecolor{d15color}{HTML}{2b2b2b}
\definecolor{bus1color}{HTML}{2b2b2b}
\definecolor{bus2color}{HTML}{2b2b2b}
\definecolor{bus3color}{HTML}{2b2b2b}
\definecolor{bus4color}{HTML}{2b2b2b} 

% Definitions
\def\refdist{2.5mm} % Channel ref arrow length
\def\tipangleA{125} % Tip angle for CH1
\def\tipangleB{125} % Tip angle for CH2
\def\tipangleC{-45} % Tip angle for CH3
\def\tipangleD{75} % Tip angle for CH4
\def\tipangleE{75} % Tip angle for CH5
\def\tipdist{5mm} % Tip arrow legth

% Defines axis aspect ratio
\def \axisheight {\linewidth}
\def \axiswidth {1.3\linewidth} 

\def \varchA {$v_{A}$} % Variable at channel A (1);
\def \varchB {$v_{B}$} % Variable at channel B (2);
\def \varchC {$v_{C}$} % Variable at channel C (3);
\def \varchD {$v_{D}$} % Variable at channel D (4);
\def \varchE {$v_{E}$} % Variable at channel E (5); ...

\pgfplotsset{every axis legend/.style={
		cells={anchor=center},% Centered entries
		inner xsep=3pt,inner ysep=2pt,
		nodes={inner sep=10pt,text depth=0.15em},
		anchor=north west,
		shape=rectangle,
		fill=white,
		draw=white,
		rounded corners,
		line width=1.5pt,
		at={(1.01,1)}
	}
}


\begin{document}
	\fontsize{10.5}{12}\selectfont
 \begin{tikzpicture}   
      \begin{axis}[
        height=\axisheight, % Scale the plot to \linewidth
        width=\axiswidth, % Scale the plot to \linewidth
        grid=major, % Display a grid
        minor tick num=5,
        grid style={solid,gray!25}, % Set the style
        xmin=-5,xmax=5,ymin=-5,ymax=5,          
        xticklabels=\empty,yticklabels=\empty, % End of preamble tex file
xlabel=Time: \SI{4.00}{\milli\second}/div,
] % End of axis configurations

% Settings for siunitx package
\sisetup{scientific-notation = fixed, fixed-exponent = 0, round-mode = places,round-precision = 2,output-decimal-marker = {.}}

% Arrows and labels for channel ch1
\addplot[solid,ch1color]table[x=time,y=ch1,col sep=comma]{tek0047.csv}; % Add plot data
\node[coordinate,pin={[pin distance=\refdist,pin edge={stealth-,semithick,ch1color}]0:{}}] at (axis cs:-5,-3.68){}; % Print ref arrow
\node[coordinate,pin={[pin distance=\tipdist,pin edge={stealth-,semithick,black}]-5:{CH1}}] at (axis cs:-1.924,-4.1108){}; % Print curve tip
\addlegendentry[align=center]{$v_{c1}$ @ CH1\\ \SI{500}{\V}/div\\RMS: \SI{277.1089}{\V}}

% Arrows and labels for channel ch2
\addplot[solid,ch2color]table[x=time,y=ch2,col sep=comma]{tek0047.csv}; % Add plot data
\node[coordinate,pin={[pin distance=\refdist,pin edge={stealth-,semithick,ch2color}]0:{}}] at (axis cs:-5,-0.22){}; % Print ref arrow
\node[coordinate,pin={[pin distance=\tipdist,pin edge={stealth-,semithick,black}]-10:{CH2}}] at (axis cs:-1.947,-0.58309){}; % Print curve tip
\addlegendentry[align=center]{$v_{c2}$ @ CH2\\ \SI{500}{\V}/div\\RMS: \SI{271.1974}{\V}}

% Arrows and labels for channel ch3
\addplot[solid,ch3color]table[x=time,y=ch3,col sep=comma]{tek0047.csv}; % Add plot data
\node[coordinate,pin={[pin distance=\refdist,pin edge={stealth-,semithick,ch3color}]0:{}}] at (axis cs:-5,-2.02){}; % Print ref arrow
\node[coordinate,pin={[pin distance=\tipdist,pin edge={stealth-,semithick,black}]-4:{CH3}}] at (axis cs:-1.97,-2.3615){}; % Print curve tip
\addlegendentry[align=center]{$v_{c3}$ @ CH3\\ \SI{500}{\V}/div\\RMS: \SI{272.9279}{\V}}

% Arrows and labels for channel math
\addplot[solid,mathcolor]table[x=time,y=math,col sep=comma]{tek0047.csv}; % Add plot data
\node[coordinate,pin={[pin distance=\refdist,pin edge={stealth-,semithick,mathcolor}]0:{}}] at (axis cs:-5,2.58){}; % Print ref arrow
\node[coordinate,pin={[pin distance=\tipdist,pin edge={stealth-,semithick,black}]128:{MATH}}] at (axis cs:-1.97,2.9155){}; % Print curve tip
\addlegendentry[align=center]{$\frac{v_{c1}+v_{c2}+v_{c3}}{3}$ @ MATH\\ \SI{200}{\V}/div}
      \end{axis}   
   
      \begin{axis}[ % Just to plot x  and y axis in the middle of figure
		height=\axisheight, % Scale the plot to \linewidth
        width=\axiswidth, % Scale the plot to \linewidth
      	grid=none, minor tick num=5,
      	xmin=-5,xmax=5,ymin=-5,ymax=5,   % Axis Limits
      	axis x line=middle,axis y line=middle,
      	xticklabels=\empty,yticklabels=\empty,
      	]        	
      \end{axis}             
    \end{tikzpicture}
\end{document} % End of file